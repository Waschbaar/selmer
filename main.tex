\documentclass{article}

\usepackage{amsmath}
\usepackage{amsthm}
\usepackage{amssymb}
\usepackage{amsfonts}
\usepackage{mathrsfs}
\usepackage{graphicx}
\usepackage{float}
\usepackage{wrapfig}
\usepackage{tikz}
\usepackage{tikz-cd}

\usepackage[left=2cm,right=2cm,top=3cm,bottom=3cm]{geometry}

\newtheorem{thm}{Theorem}
\newtheorem{defi}[thm]{Definition}
\newtheorem{cor}[thm]{Corollary}
\newtheorem{rmk}[thm]{Remark}
\newtheorem{prop}[thm]{Proposition}
\newtheorem{lem}[thm]{Lemma}

\newcommand{\ord}{\mathcal{O}}
\newcommand{\mn}{\mu_{n}}
\newcommand{\kmkn}{K ^{\times} / (K ^{\times})^{n}}
\newcommand{\kvmkn}[1]{(#1) ^{\times} / ((#1) ^{\times})^{n}}

\DeclareMathOperator{\hhom}{Hom}
\DeclareMathOperator{\spec}{Specm}

\title{Selmer groups are finite}
\author{Xiaomin Chu}
\date{Nov 2023}

\begin{document}
\maketitle

\paragraph{Notation}
For a group $ G $, denote $ G [n] $ the $ n $-torsion of $ G $.
We distinguish between canonical and non-canonical isomorphisms in the same way as Milne does.
But I write $ = $ for canonical isomorphisms, and $ \cong $ for non-canonical ones.
This is slightly abusive since $ = $ doen't neccessarily mean up to contractible choice...

\paragraph{Setup}
$ K $ a number field, $ E $ an ellipic curve over $ K $.
Fix an algebraic closure $ \overline{K} $ of $ K $,
and algebraic extensions of $ K $ means a subfield of $ \overline{K} $.
$ \spec \ord _{K} $ denotes the set of finite places of $ K $.
Denote $ K _{v} $ the completion of $ K $ at $ v $.
For each $ v\in \spec \ord _{K} $, choose a place
$ \overline{v}\in \spec \ord _{\overline{K}} $ extending $ v $,
and identify the absolute Galois group of $ K _{v} $ with
the decomposition group of $ \overline{v}/v $.
Let $ K _{v}^{nr} $ denote the maximal unramified extension of $ K _{v} $
inside $ \overline{K}_{\overline{v}} $.
Fix $ n \geq 2 $ a positive integer.

Recall that in order to prove the weak Mordell-Weil conjecture,
we intruduced the Selmer groups
$$ \mathrm{Sel}^{(n)}(E) = \ker H ^{1}(K, E [n])\to
\prod _{v\in \spec \ord _{K}} H ^{1}(K _{v}, E (\overline{K _{v}})) $$
and wished that it's finite.

\section{For $ \mathbb{G}_{m} $}

We start with something much simpler than an elliptic curve, i.e. $ \mathbb{G}_{m} $.
Note that $ \mathbb{G}_{m}(K) = K ^{\times} $ is of course not finitely generated.
(In fact here we should consider $ \mathbb{G}_{m}(\ord _{K}) $, which is finitely generated)

Consider the short exact sequence
$$ 0 \to \mn \to \overline{K}^{\times} \xrightarrow{n} \overline{K}^{\times}\to 0 $$
We may extract the following exact sequence from the cohomological long exact sequence
$$ 0\to \kmkn \to H ^{1}(K, \mn)\to H ^{1}(K, \overline{K}^{\times})[n]\to 0  $$
But Hilbert 90 says that $ H ^{1}(K, \overline{K}^{\times}) = 0 $,
so actually $ \kmkn = H ^{1}(K, \mn) $.

$ \kmkn $ is approximately $ \bigoplus _{v\in\spec \ord _{K}} \mathbb{Z}v $.
To make it precise, we use the celebrated exact sequence in algebraic number theory
$$ 0\to \ord _{K}^{\times} \to K ^{\times}\to \bigoplus _{v\in \spec \ord _{K}} \mathbb{Z}v 
\to \mathrm{Cl}(K)\to 0$$
where $ \mathrm{Cl}(K) $, the ideal class group of $ K $, is finite.
Applying the snake lemma to
\begin{figure}[H]
\centering
\begin{tikzcd}
\quad  
&K ^{\times} \arrow[r] \arrow[d, "n"]
&\bigoplus _{v\in\spec \ord _{K}} \mathbb{Z}v \arrow[r] \arrow[d, "n"]
&\mathrm{Cl}(K) \arrow[r] \arrow[d, "n"]
& 0\\
0 \arrow[r]
&K ^{\times}/\ord _{K}^{\times} \arrow[r] 
&\bigoplus _{v\in\spec \ord _{K}} \mathbb{Z}v \arrow[r]
&\mathrm{Cl}(K)
& \quad
\end{tikzcd}
\end{figure}
gives us the exact sequence
$$ \mathrm{Cl}(K)[n] \to K ^{\times}/(\ord _{K}^{\times} (K ^{\times})^{n})\to
\bigoplus _{v\in \spec \ord _{K}} (\mathbb{Z}/n \mathbb{Z}) v \to
\mathrm{Cl}(K)/n \mathrm{Cl}(K)$$

But Dirichlet's unit theorem says that
$ \ord _{K}^{\times} $ is a finitely generated abelian group,
so $ \kmkn $ isn't much different from $ K ^{\times}/(\ord _{K}^{\times} (K ^{\times})^{n}) $.
To be more precise, we have the following exact sequence
$$ 0\to \ord _{K}^{\times}/ (\ord _{K}^{\times})^{n} \to
\kmkn \to K ^{\times}/(\ord _{K}^{\times} (K ^{\times})^{n})\to 0$$

To summurize the discussion above, we have
\begin{thm}
There are two finite groups $ R _{1}(K, n), R _{2}(K, n) $ such that
$$ R _{1}(K, n)\to \kmkn \to  
\bigoplus _{v\in \spec \ord _{K}} (\mathbb{Z}/n \mathbb{Z}) v \to R _{2}(K, n) $$
is exact.\label{comparison:h1}
\end{thm}
In particular,
\begin{cor}
If $ S $ is a subgroup of $ \kmkn $ whose image in
$ \bigoplus _{v\in \spec \ord _{K}} (\mathbb{Z}/n \mathbb{Z}) v$ is finite,
then $ S $ is finite. \label{subgrp:finite}
\end{cor}

\begin{proof}
$ S $ is an extension of its image in
$ \bigoplus _{v\in \spec \ord _{K}} (\mathbb{Z}/n \mathbb{Z}) v$ and a subgroup of
$ R _{1}(K, n) $.
\end{proof}

\subsection{Aside: Selmer groups for $ \mathbb{G}_{m} $}

We may define the Selmer groups for $ \mathbb{G}_{m} $ as well.
Here we need to consider its points over $ \ord _{K} $ instead of $ K $.
In fact we're also considering the points over $ \ord _{K} $ for elliptic curves,
but elliptic curves are projective so we can rescale the coordinates...

Consider the exact sequence
$$ 0\to \mn\to \ord _{\overline{K}}^{\times}\xrightarrow{n} \ord _{\overline{K}}^{\times}\to 0 $$
Again extract from the cohomological long exact sequence the following
$$ 0\to \ord _{K}^{\times}/(\ord _{K}^{\times})^{n}
\to H ^{1}(K, \mn)\to H ^{1}(K, \ord _{\overline{K}}^{\times})[n]\to 0$$

Define $ \mathrm{Sel}^{n}(\mathbb{G}_{m}/\ord _{K}) $ to be 
$$ \mathrm{Sel}^{n}(\mathbb{G}_{m}/\ord _{K}) =
\ker H ^{1}(K, \mn)\to \prod _{v\in\spec \ord _{K}} H ^{1}(K _{v}, \ord _{\overline{K}_{v}}^{\times}) $$

We can prove that $ \mathrm{Sel}^{n}(\mathbb{G}_{m}/\ord _{K}) $ is finite using the method that's
going to be applied to elliptic curves later.
The proof is listed here as an illustration of the method.
\begin{thm}
$ \mathrm{Sel}^{n}(\mathbb{G}_{m}/\ord _{K}) $ is finite.
\end{thm}

\begin{proof}
Let $ \gamma\in \mathrm{Sel}^{n}(\mathbb{G}_{m}/\ord _{K}) $ corresponds to
$ \alpha\in \kmkn $ under the (canonial) isomorphism $ H ^{1}(K, \mn)\to \kmkn $.
Then for all $ v\in \spec\ord _{K} $, there is an element $ \beta\in
\ord _{K}^{\times}/(\ord _{K}^{\times})^{n}$ mapping to $ \alpha $.

If the residue characteristic of $ v $ doesn't divide $ n $,
then Hensel's lemma gives a $ \xi\in \ord _{K _{v}^{nr}}^{\times} $ such that
$ \xi ^{n} = \beta $.
Then $ \alpha $ is mapped to the image of $ \xi ^{n} = 0 $ upon further restricting to
$ \kvmkn{K _{v}^{nr}} $.
But the restriction maps are natural inclusions under the isomorphism,
so $ \alpha $ is a $ n $-th power in $ K _{v}^{nr} $.
So actually $ n\mid v (\alpha)$.
Since this is true for all $ v\nmid n $,
the image of $ \mathrm{Sel}^{n}(\mathbb{G}_{m}/\ord _{K}) $,
under the map $ \kmkn \to \bigoplus _{v\in \spec \ord _{K}} (\mathbb{Z}/n \mathbb{Z})v $,
lies in the subgroup
$ \bigoplus _{v\mid n} (\mathbb{Z}/n \mathbb{Z})v $, which is finite.
By Corollary \ref{subgrp:finite}, $ \mathrm{Sel}^{n}(\mathbb{G}_{m}/\ord _{K}) $ is finite. 
\end{proof}

Note however that we can't reprove Dirichlet's unit theorem from this perspective
since we used it in proving Theorem \ref{comparison:h1}.

\section{Reducing to $ E [n]\subset E (K) $}

In general $ E $ is very different from $ \mathbb{G}_{m}\times \mathbb{G}_{m} $.
But we can compare their $ H ^{1} $ somehow forcefully.
Suppose that $ E [n] \subset E (K) $, then $ G _{K} $ acts trivially on $ E [n] $.
Then $ H ^{1}(K, E [n]) \cong \hhom (G _{K}, \mathbb{Z}/n \mathbb{Z} \times \mathbb{Z}/n \mathbb{Z})$
by choosing a basis $ (a, b) $ of $ E [n] $.
In this case, Weil pairing implies that $ \mn \subset K ^{\times} $.
So we have
$$ H ^{1}(K, E [n]) \cong \hhom (G _{K}, \mathbb{Z}/n \mathbb{Z})\times
\hhom (G _{K}, \mathbb{Z}/n \mathbb{Z})\cong H ^{1}(K, \mn)\times H ^{1}(K, \mn)
$$
Which then equals $ \kmkn\times \kmkn $.
This is very nice.

Now we reduce to this situation.
\begin{prop}
Suppose $ L/K $ is a finite Galois extension.
If $ \mathrm{Sel}^{n}(E/L) $ is finite, then
$ \mathrm{Sel}^{n}(E/K) $ is also finite.
\end{prop}

\begin{proof}
We have the inflation-restriction exact sequence
$$ 0\to H ^{1}(G _{L/K}, E [n](L))\xrightarrow{\mathrm{Inf}}
H ^{1}(K, E [n]) \xrightarrow{\mathrm{Res}}
H ^{1}(L, E [n])$$
and $ H ^{1}(G _{L/K}, E [n](L)) $ is clearly finite.
So $ \mathrm{res}: \mathrm{Sel}^{n}(E/K)\to \mathrm{Sel}^{n}(E/L)^{G _{L/K}} $
has finite kernel.
\end{proof}

So if we can prove that $ \mathrm{Sel}^{n}(E/K (E [n])) $ is finite,
then we have $ \mathrm{Sel}^{n}(E/K) $ is finite.
From now on we assume that $ E [n]\subset E (K) $.
\begin{rmk}
If you don't like the Weil pairing, you can take $ \mn \subset K ^{\times} $ as an
additional assumption that doesn't hurt.
\end{rmk}

\section{Solving locally}

You might have been asking,
"why the f**k do we care about the Selmer groups"?
The point is that, solving equations in $ K _{v} $ is much easier that
solving in $ K $.
We might hope that some point $ Q \in E (K)$ to be in $ nE (K _{v}) $ when it's not in $ n E (K) $.

Recall that we have the commutative diagram in which the rows are exact
\begin{figure}[H]
\centering
\begin{tikzcd}
0 \arrow[r]
& E(K)/nE(K) \arrow[d, "\mathrm{res}"] \arrow[r]
& H^1 (K, E[n]) \arrow[d, "\mathrm{res}"]\\
0 \arrow[r]
& E(K_v)/nE(K_v) \arrow[r]
& H^1 (K_v, E[n])
\end{tikzcd}

\end{figure}

Now suppose that $ \gamma\in \mathrm{Sel}^{n}(E/K) $.
Then $ \mathrm{res}(\gamma) $ lies in the image of $ E (K _{v})/n E (K _{v}) $.
Say it's the image of $ Q\in E (K _{v}) $.

A very nice property of the local field is that,
although $ Q $ is not neccessarily in $ n E (K _{v}) $,
the extension $ K _{v}\left(\frac{Q}{n}\right) $ is unramified over $ K _{v} $
under some genericity condition on $ v $.
More precisely,

\begin{thm}
There is a finite $ S\subset \spec \ord _{K} $ such that
if $ v\not\in S $ and $ Q\in E (K _{v}) $,
then $ Q\in n E (K _{v}^{nr}) $.
\end{thm}

This is essentially Hensel's lemma combined with the fact that
if $ k $ is any algebraically closed field and
$ E/k $ an elliptic curve,
then $ n: E (k)\to E (k) $ is surjective.
Now we work rigorously.

For the moment let $ F $ be a 
complete nonarchimedean discretely valued field of characteristic $ 0 $,
with a perfect residue field $ k $ of characteristic $ p $.
%We start with a version of Hensel's lemma
%\begin{prop}
%Let $ G _{1},\ldots,G _{r}\in \ord _{F}[x _{1}, \ldots, x _{r}] $ and
%$ y _{1},\ldots, y _{r}\in k $ be a set of solutions of
%$ \overline{G _{1}},\ldots,\overline{G _{r}} $.
%If $ \ \det \mathrm{Jab}(G _{i})(y _{i}) \neq 0 $ in $ k $,
%then there are uniquely determined $ z _{1},\ldots,z _{r}\in \ord _{F} $
%such that $ z _{i} $ maps to $ y _{i} $ under reduction modulo $ \mathfrak{m}_{F} $,
%and $ G _{i}(z _{1},\ldots,z _{r})=0 $ for all $ i $.
%\end{prop}
%The proof that I'm not going to give here is by Newton iteration. The assumption on the
%Jacobian matrix ensures that the iteration is possible,
%and it's always convergent since we're in nonarchimedean situation.

\begin{prop}
Suppose that $ E $ has good reduction modulo $ \mathfrak{m}_{F} $,
and $ p\nmid n $.
Then the multiplication by $ n $ map on $ E (F ^{nr}) $ is surjective.
\end{prop}

\begin{proof}
We have the following commutative diagram with exact rows
\begin{figure}[H]
\centering
\begin{tikzcd}
0 \arrow[r]
& E^1(F^{nr}) \arrow[r] \arrow[d, "n"]
& E(F^{nr}) \arrow[r] \arrow[d, "n"]
& E(\overline{k}) \arrow[r] \arrow[d, "n"]
& 0\\
0 \arrow [r]
& E^1(F^{nr}) \arrow[r]
& E(F^{nr}) \arrow[r]
& E(\overline{k}) \arrow[r]
& 0
\end{tikzcd}
\end{figure}
The left vertical map is an isomorphism as $ p\nmid n $,
the right vertical map is surjective since maps between curves over an algebraically closed
field is either surjective or has image a point.
Then an application of the snake lemma shows that the middle vertical map is also surjective.
\end{proof}

Now we can prove the theorem
\begin{proof}
[Proof of the theorem]
Let $ S = \{v\mid 2, v\mid 3, v\mid n, E \text{ has bad reduction at } v\} $.
Then $ S $ is finite.
If $ v\not\in S $, then we may use the above proposition to obtain that
$ n: E (K _{v}^{nr})\to E (K _{v}^{nr}) $ is surjective.
\end{proof}

\section{Now the Selmer groups}

Now we prove that Selmer groups are finite.

\begin{thm}
$ \mathrm{Sel}^{n}(E/K) $ is finite.
\end{thm}

\begin{proof}
Consider the following commutative diagram with exact rows
\begin{figure}[H]
\centering
\begin{tikzcd}
0 \arrow[r]
& E(K)/nE(K) \arrow[r] \arrow[d, "\mathrm{res}"]
& H^1(K, E[n]) \arrow[r, "\cong"] \arrow[d, "\mathrm{res}"]
& \kmkn \times \kmkn\arrow[d]\\
0 \arrow[r]
& E(K_v)/nE(K_v) \arrow[r] \arrow[d, "\mathrm{res}"]
& H^1(K_v, E[n]) \arrow[r, "\cong"] \arrow[d, "\mathrm{res}"]
& \kvmkn{K_v} \times \kvmkn{K_v}\arrow[d]\\
0 \arrow[r]
& E(K_v^{nr})/nE(K_v^{nr}) \arrow[r]
& H^1(K_v^{nr}, E[n]) \arrow[r, "\cong"]
& \kvmkn{K_v^{nr}} \times \kvmkn{K_v^{nr}}
\end{tikzcd}

\end{figure}
where $ v\not\in S $.
Let $ \gamma\in \mathrm{Sel}^{n}(E/K) $, which corresponds to $ (\gamma _{1}, \gamma _{2}) $
under the last isomorphism on the first line.
Then by assumption, there is some $ Q\in E (K _{v}) $ mapping to 
$ \mathrm{res}(\gamma)\in H ^{1}(K _{v}, E [n]) $.
But the image of $ Q $ is $ 0 $ in the third line, and so is $ \gamma $.
So in the end, we have that $ n\mid v (\gamma _{1}), n\mid v (\gamma _{2}) $.

Consider the maps
$$ t _{i} = \mathrm{Sel}^{n}(E/K) \to \kmkn \times
\kmkn \xrightarrow{\pi _{i}} \kmkn \to \biguplus _{v\in \spec \ord _{K}} (\mathbb{Z}/n \mathbb{Z})v$$
where $ i=1,2 $.
The image of $ t _{i} $ is finite since it maps into $ \bigoplus _{v\in S}
(\mathbb{Z}/n \mathbb{Z})v$.
Then we conclude that
the images of $ \mathrm{Sel}^{n}(E/K) $ in the two factors of $\kmkn\times \kmkn $ are finite.
Then $ \mathrm{Sel}^{n}(E/K) $ is finite.
\end{proof}

\end{document}


