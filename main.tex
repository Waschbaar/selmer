\documentclass{article}

\usepackage{amsmath}
\usepackage{amsthm}
\usepackage{amssymb}
\usepackage{amsfonts}
\usepackage{mathrsfs}
\usepackage{graphicx}
\usepackage{float}
\usepackage{wrapfig}
\usepackage{tikz}
\usepackage{tikz-cd}

\newtheorem{thm}{Theorem}
\newtheorem{defi}[thm]{Definition}
\newtheorem{cor}[thm]{Corollary}
\newtheorem{rmk}[thm]{Remark}
\newtheorem{prop}[thm]{Proposition}
\newtheorem{lem}[thm]{Lemma}

\newcommand{\ord}{\mathcal{O}}
\newcommand{\mn}{\mu_{n}}
\newcommand{\kmkn}{K ^{\times} / (K ^{\times})^{n}}

\DeclareMathOperator{\hhom}{Hom}
\DeclareMathOperator{\spec}{Specm}

\title{Selmer groups are finite}
\author{Xiaomin Chu}
\date{Nov 2023}

\begin{document}
\maketitle

\paragraph{Setup}
$ K $ a number field, $ E $ an ellipic curve over $ K $.
Fix an algebraic closure $ \overline{K} $ of $ K $,
and algebraic extensions of $ K $ means a subfield of $ \overline{K} $.
$ \spec \ord _{K} $ denotes the set of finite places of $ K $.
For each $ v\in \spec \ord _{K} $, choose a place
$ \overline{v}\in \spec \ord _{\overline{K}} $ extending $ v $,
and identify the absolute Galois group of $ K _{v} $ with
the decomposition group of $ \overline{v}/v $.
Fix $ n \geq 2 $ a positive integer.

Recall that in order to prove the weak Mordell-Weil conjecture,
we intruduced the Selmer groups
$$ \mathrm{Sel}^{(n)}(E) = \ker H ^{1}(K, E [n])\to
\prod _{v\in \spec \ord _{K}} H ^{1}(K _{v}, E (\overline{K _{v}})) $$
and wished that it's finite.

\section{For $ \mathbb{G}_{m} $}

We start with something much simpler than an elliptic curve, i.e. $ \mathbb{G}_{m} $.
Note that $ \mathbb{G}_{m}(K) = K ^{\times} $ is of course not finitely generated.

Consider the short exact sequence
$$ 0 \to \mn \to \overline{K}^{\times} \xrightarrow{n} \overline{K}^{\times}\to 0 $$
We may extract the following exact sequence from the cohomological long exact sequence
$$ 0\to \kmkn \to H ^{1}(K, \mn)\to H ^{1}(K, \overline{K}^{\times})[n]\to 0  $$
But Hilbert 90 says that $ H ^{1}(K, \overline{K}^{\times}) = 0 $,
so actually $ \kmkn = H ^{1}(K, \mn) $.

$ \kmkn $ is approximately $ \bigoplus _{v\in\spec \ord _{K}} \mathbb{Z}v $.
To make it precise, we use the celebrated exact sequence in algebraic number theory
$$ 0\to \ord _{K}^{\times} \to K ^{\times}\to \bigoplus _{v\in \spec \ord _{K}} \mathbb{Z}v 
\to \mathrm{Cl}(K)\to 0$$
where $ \mathrm{Cl}(K) $, the ideal class group of $ K $, is finite.
Applying the snake lemma to
\begin{figure}[H]
\centering
\begin{tikzcd}
\quad  
&K ^{\times} \arrow[r] \arrow[d, "n"]
&\bigoplus _{v\in\spec \ord _{K}} \mathbb{Z}v \arrow[r] \arrow[d, "n"]
&\mathrm{Cl}(K) \arrow[r] \arrow[d, "n"]
& 0\\
0 \arrow[r]
&K ^{\times}/\ord _{K}^{\times} \arrow[r] 
&\bigoplus _{v\in\spec \ord _{K}} \mathbb{Z}v \arrow[r]
&\mathrm{Cl}(K)
& \quad
\end{tikzcd}
\end{figure}
gives us the exact sequence
$$ \mathrm{Cl}(K)[n] \to K ^{\times}/(\ord _{K}^{\times} (K ^{\times})^{n})\to
\bigoplus _{v\in \spec \ord _{K}} (\mathbb{Z}/n \mathbb{Z}) v \to
\mathrm{Cl}(K)/n \mathrm{Cl}(K)$$

But Dirichlet's unit theorem says that
$ \ord _{K}^{\times} $ is a finitely generated abelian group,
so $ \kmkn $ isn't much different from $ K ^{\times}/(\ord _{K}^{\times} (K ^{\times})^{n}) $.
To be more precise, we have the following exact sequence
$$ 0\to \ord _{K}^{\times}/ (\ord _{K}^{\times})^{n} \to
\kmkn \to K ^{\times}/(\ord _{K}^{\times} (K ^{\times})^{n})\to 0$$

To summurize the discussion above, we have
\begin{thm}
There are two finite groups $ R _{1}(K, n), R _{2}(K, n) $ such that
$$ R _{1}(K, n)\to \kmkn \to  
\bigoplus _{v\in \spec \ord _{K}} (\mathbb{Z}/n \mathbb{Z}) v \to R _{2}(K, n) $$
is exact.
\end{thm}
In particular,
\begin{cor}
If $ S $ is a subgroup of $ \kmkn $ whose image in
$ \bigoplus _{v\in \spec \ord _{K}} (\mathbb{Z}/n \mathbb{Z}) v$ is finite,
then $ S $ is finite.
\end{cor}

\begin{proof}
$ S $ is an extension of its image in
$ \bigoplus _{v\in \spec \ord _{K}} (\mathbb{Z}/n \mathbb{Z}) v$ and a subgroup of
$ R _{1}(K, n) $.
\end{proof}

\section{Reducing to $ E [n]\subset E (K) $}

In general $ E $ is very different from $ \mathbb{G}_{m}\times \mathbb{G}_{m} $.
But we can compare their $ H ^{1} $ somehow forcefully.
Suppose that $ E [n] \subset E (K) $, then $ G _{K} $ acts trivially on $ E [n] $.
Then $ H ^{1}(K, E [n]) \cong \hhom (G _{K}, \mathbb{Z}/n \mathbb{Z} \times \mathbb{Z}/n \mathbb{Z})$
by choosing a basis $ (a, b) $ of $ E [n] $.
In this case, Weil pairing implies that $ \mn \subset K ^{\times} $.
So we have
$$ H ^{1}(K, E [n]) \cong \hhom (G _{K}, \mathbb{Z}/n \mathbb{Z})\times
\hhom (G _{K}, \mathbb{Z}/n \mathbb{Z})\cong H ^{1}(K, \mn)\times H ^{1}(K, \mn)
$$
Which then equals $ \kmkn\times \kmkn $.
This is very nice.

Now we reduce to this situation.
\begin{prop}
Suppose $ L/K $ is a finite Galois extension.
If $ \mathrm{Sel}^{n}(E/L) $ is finite, then
$ \mathrm{Sel}^{n}(E/K) $ is also finite.
\end{prop}

\begin{proof}
We have the inflation-restriction exact sequence
$$ 0\to H ^{1}(G _{L/K}, E [n](L))\xrightarrow{\mathrm{Inf}}
H ^{1}(K, E [n]) \xrightarrow{\mathrm{Res}}
H ^{1}(L, E [n])$$
and $ H ^{1}(G _{L/K}, E [n](L)) $ is clearly finite.
So $ \mathrm{res}: \mathrm{Sel}^{n}(E/K)\to \mathrm{Sel}^{n}(E/L)^{G _{L/K}} $
has finite kernel.
\end{proof}

So if we can prove that $ \mathrm{Sel}^{n}(E/K (E [n])) $ is finite,
then we have $ \mathrm{Sel}^{n}(E/K) $ is finite.
From now on we assume that $ E [n]\subset E (K) $.
\begin{rmk}
If you don't like Weil pairing, you can take $ \mn \subset K ^{\times} $ as an
additional assumption that doesn't hurt.
\end{rmk}

\section{Proof}

\end{document}


